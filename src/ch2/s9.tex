\wde{2.9.1 Stationary} A distribution $\pi = (\pi_1,\pi_2,\cdots) \ge 0$ is stationary if it satisfies the global balance equations, i.e. $\pi = \pi P$, and $\sum_{j \in S} = \pi_j = 1$. A chain with a stationary distribution is said to be in a stationary state or a steady state.
\wt{Tut Wk 5 Detailed Balance Eq.} A chain satisfies the detailed balance condition if $\pi_i p_{ij} = \pi_j p_{ji}$. If the detailed balance condition is satisfies the global balance condition is satisfied, so the chain is stationary.
\wpr{2.9.2} If a chain is initially in a stationary distribution, $a^{(0)} = \pi$, then $a^{(n)} = \pi$ for all $n \ge 0$.
\wt{2.9.3} (a) For aperiodic, irreducible chains the limiting probabilities are independent of the initial state, that is, for every $i, j \in S$, $\lim_{n \to \infty} p^{(n)}_{jj} = \lim_{n \to \infty} p^{(n)}_{ij}$ and we call this limit $\pi_j$. (b) If the chain is also positive recurrent then the limiting probability distribution $\pi = (\ldots, \pi_j, \ldots)$ is the unique stationary distribution.