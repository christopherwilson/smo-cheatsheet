\subsubsection*{Superpositioning \& Splitting}
\wt{Superpositioning 3.3.1} Let $(N^{(i)}_t)_{t \ge 0}$ be $PP(\lambda_i)$ for each $i = 1, \ldots, k$. Then, with $N_t = N^{(1)}_t + \cdots + N^{(k)}_t$, $(N_t)_{t \ge 0}$ is a $PP(\lambda_1 + \cdots + \lambda_k)$.
\wt{Splitting 3.3.2} Suppose $(N_t)_{t \ge 0}$ is a $PP(\lambda)$. If an event in this process is of type $i$ with probability $p_i$ independent of the other events, where $\sum^k_{i=1} p_i = 1$, then the processes $(N^{(1)}_t)_{t \ge 0}, (N^{(2)}_t)_{t \ge 0}, \ldots, (N^{(k)}_t)_{t \ge 0}$ are independent Poisson processes with rates $\lambda p_1, \lambda p_2, \ldots, \lambda p_k$ respectively.