\subsubsection*{Basic Definitions}
\wde{Markov Property 2.1.1} A stochastic process is said to have Markov property if given the present state, the future events are independent of the past. For the discrete time discrete state space processes $(X_n)_{n \in \mathbb{N}}$ this property can be stated as $P(X_{n+1} = j | X_{n} = i, X_{n-1} = i_{n-1}m \ldots, X_{0} = i_{0}) = P(X_{n+1} = j | X_{n}=i)$ $\forall$ $k,i,i_{n-1},\ldots,i_0 \in S$ and $n \in \N$, and we also define $p_{ij}(n) \equiv P(X_{n+1} = j | X_n = {i})$ and refer to it as the (one step) transition probability from $i$ to $j$ at time $n$.
\wl{$p_{ij}$ Properties 2.1.X} $p_{ij} \ge 0$, for all $i,j \in S$, and $\sum_{j \in S} p_{ij}(n) = 1$.