\subsubsection*{Classification of States}
\wde{Accessibility 2.5.1} A state $j$ is said to be accessible from state $i$, denoted $i \to j$, if there exists and $n \ge 0$ such that $p^{(n)}_{ij} > 0$.
\wde{Communication 2.5.2} Two sates $i$ and $j$ are said to communicate, denoted $i \comn j$, if and only if $i \to j$ and $j \to i$.
\wt{2.5.3} Communication is an equivalence relation, that is: 
(1) $i \comn i\ \forall i \in S$ (reflexive); 
(2) If $i \comn j$ then $j \comn i$ (symmetric); 
(3) If $i \comn j$ and $j \comn k$ then $i \comn k$ (transitive).
\wde{Communication Class 2.5.4} Let $C$ be a subset of the state space $S$. $C$ is said to be a communicating class if: 
(1) $i,j \in C$ then $i \comn j$; 
(2) $i \in C$ and $i \comn j$ then $j \in C$. 
If in addition to these properties we cannot leave $C$, that is for any $i \in C$ and any $k \notin C$ we have $i \nrightarrow k$, then $C$ is said to be a closed communicating class.
\wde{(Ir)Reducible 2.5.5} A DTMC is said to be irreducible if the state space $S$ is a single (closed) communicating class, and it is called reducible if it is composed of several communicating classes.