\subsubsection*{Conditional Probability}
\wde{Conditional Probability 1.1.2} The conditional probability of event $A$ given $B$ (s.t. $P(B) > 0$) has occurred, denoted $P(A|B)$, is defined as $P(A|B) \equiv \frac{P(A\cap B)}{P(B)}$.
\wt{Law of Total Probability 1.1.4} For events $B_1, B_2, \ldots, B_n$ (all with non-zero probabilities) that partition the sample space $\Omega$ (that is $B_i \cap B_j = \emptyset$ for $i \ne j$, and $\bigcup^{n}_{i=1}B_i = \Omega$) and for any event $A$: 
$P(A) = \sum^{n}_{i=1}P(A|B_i)P(B_i)$.
\wt{1.1.8} 
$X,Y$ discrete: $p(x) = \sum_y p(x|Y=y)p(y)$; 
$X$ discrete, $Y$ continuous: $p(x) = \int p(x|Y=y)f(y)\ dy$; 
$X$ continuous, $Y$ discrete: $f(x) = \sum_y f(x|Y=y)p(y)$; 
$X,Y$ continuous: $f(x) = \int f(x|Y=y)f(y)\ dy$.